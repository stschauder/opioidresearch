\documentclass{article}
\usepackage[utf8]{inputenc}

\usepackage{graphicx}
\graphicspath{ {images/} }

\usepackage{natbib}


\title{The Effect of the Medicaid Expansion on Opioid Addiction}
\author{Stephanie Schauder}
\date{August 2017}

\begin{document}

\maketitle


%use F1 + fn to compile

Example citation: Here we use data from \citep{Courtemanche2017}.  

\section{Abstract}

In this paper, I examine the effect of an increase in insurance coverage, by way of the Medicaid expansion, on opioid addiction.  I hypothesize that expanding insurance coverage could potentially improve indicators of opioid addiction if people are able to receive preventative care that reduces chronic pain or if they are able to enter addiction treatment.  Alternatively, expanding insurance could worsen the epidemic if more people are seeking medical assistance and pain medicine now that it is covered by insurance.  My outcome variables are deaths due to opioid overdose, opioid addiction treatment admissions, and opioids sold.



\section{Introduction}
Opioid addiction in the United States is a serious problem that has been growing since the late 1990?s.  The rate of overdose deaths related to opioids has increased by 200\% since the year 2000, and opioid overdoses are now the primary reason for the rapid increase in deaths due to drug poisoning in the past 20 years.  In 2014 there were more deaths from overdoses of any kind than from car accidents (Rudd et al. 2016).  

Opioids however, are not new to the United States for either medical or recreational purposes.  The first opioid epidemic took place in the late 1800?s and occurred for a variety of reasons, most notably, the treatment of pain.  Opioids were primarily used for infections and chronic conditions due to lack of other medical options (Kolodny et al. 2015).  However, the epidemic subsided as medical advances reduced such diseases as diarrhea, and provided better prescription drug options (Kolodny et al. 2015).  

For the majority of the 20th century, the medical paradigm was that opioids should be prescribed minimally for pain, and should only be given more liberally in the case of terminal illness (Meldrum 2016).  However, this way of thinking began to change in the 1980?s as doctors recognized that chronic pain was severely undertreated.  This shift is perhaps best illustrated by the 1995 campaign to recognize pain as ?the fifth vital sign,? which encouraged physicians to treat pain more aggressively (Kolodny et al. 2015).  Coinciding with this renewed interest in addressing pain, was the release and heavy marketing of OxyContin (extended release oxycodone) and research claiming that addiction to opioid medication was simply physical dependence (Kolodny et al. 2015).  As a result, the number of opioids prescribed for pain has increased fourfold since 1999 (Centers for Disease Control and Prevention 2016).   

The Affordable Care Act (ACA) was implemented in 2010. Originally, the ACA included a provision to expand Medicaid to all people under 138\% of the poverty line in all states.  However, a supreme court decision in 2012 ruled that states could decide if they wanted to adopt the Medicaid expansion.  As of January 1, 2017, 32 states including Washington D.C. have adopted the expansion, and 19 states have not (Kaiser Family Foundation 2017).  

In this paper, I will explore how the Medicaid expansion impacted the opioid epidemic.  All Medicaid programs cover at least one of the drugs used to treat opioid addiction (?Insurance and Payments? 2015).  Because the Affordable Care Act expanded Medicaid insurance coverage, people who did not previously have insurance, but were able to obtain insurance after the expansion may have been able to get addiction treatment, which could lead to a reduction in deaths.  Additionally, if more citizens have health insurance, they may be able to treat injuries or conditions before they reach the point where opioids are necessary.  However, because the Medicaid expansion increased the number of people who were able to see doctors and receive treatment, the ACA may have made it possible for more people to be proscribed opioids and subsequently become addicted.  Therefore, the effect of the ACA on opioid deaths is unclear.

Several studies have found that the ACA Medicaid expansion increased the number of people receiving health insurance.  Sommers et al. look at self-reported insurance coverage and find that significantly more adults were able to obtain coverage as a result of the Medicaid expansion (Benjamin D. Sommers et al. 2015).  Courtemanche et al. find that the number of people insured increased in both expansion states and non-expansion states but that it was significantly higher in expansion states.  Also the authors find no evidence that the Medicaid expansion was crowding out private insurance (Courtemanche et al. 2017).  Frean et al. find that the increase in insurance coverage in expansion states is not only due to people who were previously ineligible becoming eligible, but also due to previously eligible people obtaining insurance (Frean, Gruber, and Sommers 2016).  Additional studies find that more people were able to obtain insurance as a result of the ACA expansion (B. D. Sommers, Kenney, and Epstein 2014; Wherry and Miller 2016).  

In a working paper, Maclean and Saloner (2017) look at the effect of the Medicaid expansion on insurance coverage and substance use disorder (SUD) treatment.  The authors find that the number of people paying for treatment with Medicaid increased by 57 $\%$. They used Medicaid State Drug Utilization Data and found that the number of Medicaid-reimbursed prescriptions written for medicines that treat drug addiction increased by 33$\%$.  They found no effect on drug treatment admissions (Maclean and Saloner 2017). 

Other studies have shown that an increase in Medicaid enrollment has led to an increase in the utilization of medical services.  The Oregon experiment found that when Medicaid was expanded by lottery, there was an increase in the use of services for those newly covered by Medicaid (Baicker et al. 2013).  Also, Simon et al. look at the effect of the ACA Medicaid expansion on the utilization of preventative care services.  They find that the expansion not only increased insurance coverage, but also led to an increased proportion of people with a personal doctor, receiving HIV tests, and visiting the dentist.  Additionally, they observed improvements in self-reported health (Simon, Soni, and Cawley 2016).  

I will use a difference in differences approach to look at the effect of the Medicaid expansion on indicators of opioid addiction.  This paper contributes to the literature on the effect of the Medicaid expansion by looking specifically at opioid addiction.  Section 2 outlines the data that I will use.  Section 3 is a discussion of the methods and the assumptions necessary to employ them.  Section 4 presents the results, and section 5 concludes.   


\section{Data}
To assess the impact of the ACA on opioid addiction, I use a panel dataset where each observation corresponds to a particular state and year.  The outcome of interest is either overdose deaths or opioids sold.  I have data from 2003-2015 for overdose deaths and from 2008-2015 for opioids sold. 

The data on overdose deaths comes from the Centers for Disease Control and Prevention (CDC) multiple cause of death files (Centers for Disease Control and Prevention 2017).  I am interested in drug overdose deaths, and within that category, all deaths due to opioids.  One challenge with this data is that for a particular state and year, if the number of deaths that fall into a specific ICD category sum to less than 10, the number of deaths is not reported.  To construct my variable for total deaths due to opioids, I sum 9 different ICD categories.  If the number of deaths was not reported for a given state-year in any of these categories, I counted it as zero.  Because of this limitation, I also performed the analysis with overall overdose deaths (which did not suffer from this problem), and I report the results for both below.  I converted the two outcome variables, opioid related deaths and total overdose deaths, to per capita terms.  

The data on the number of drugs sold comes from the Automated Reports and Consolidated Ordering System (ARCOS).  The quantity of each medication sold is reported in grams.  This is not meaningful because it is hard to compare sales of one drug to another to determine if there was substitution.  Therefore, I focused on the opioid drugs which are commonly discussed in the literature and have well documented morphine equivalent dosages (Gordon et al. 1999).  These drugs were morphine, hydromorphone, oxymorphone, codeine, hydrocodone, meperidine, methadone, and fentanyl, and I followed the equianalgesic tables from Gordon et al. (1999).  All of the above drugs with the exception of fentanyl can be converted to oral morphine.  I created a variable which is the sum of the morphine equivalent dosages of these drugs divided by the population, which I call total opioids sold.  Fentanyl behaves more closely to intravenous morphine so it cannot be equated to oral morphine.  Therefore, I created a separate outcome variable for fentanyl per capita.  I also look separately at oxycodone per capita because this drug, the generic of OxyContin, has received much attention in the media and the introduction of OxyContin coincided with the beginning of the opioid epidemic (Kolodny et al. 2015).  Methadone and buprenorphine are the two drugs in my sample that can be used to treat opioid addiction.  However, methadone can also be prescribed for pain whereas buprenorphine is usually not.  Therefore I include an outcome variable for buprenorphine per capita as an indicator of the quantity of medications prescribed to treat opioid addiction. 

One limitation of the ARCOS data is that the dataset for the year 2012 is missing several of the important opioid drugs. I have excluded this year when I created the variable, sum of opioids, which sums the morphine equivalent dosages of the most common opioids.  

Data on the status of the Medicaid expansion comes from the Kaiser Family Foundation (Kaiser Family Foundation 2017).  Currently, 32 states and the District of Columbia have participated in the Medicaid expansion.  Of those 32, 26 officially expanded on January 1, 2014 and 6 states expanded at some point after that. Eleven states partially expanded Medicaid before 2014, but these expansions occurred to differing extents.  In 2014 all states which choose to participate (including the early partial expanders), expanded to cover adults under 138 $\%$ of the poverty line.  In this analysis documented early expanders as expanding in 2014 (B. D. Sommers, Kenney, and Epstein 2014).  As a robustness check, I report additional regressions where I exclude those states.  Figure 1 shows the 32 states that have expanded Medicaid as of 2016 (Kaiser Family Foundation 2017).  The expansion occurred across geographic and political lines; however, there is a definite pattern of states in the Southeast not adopting and states in the Northeast adopting.    

Data on Prescription Drug Monitoring Programs comes from the Prescription Drug Abuse Policy System (PDAPS). There is mixed evidence that the presence of a PDMP prevents drug abuse (Patrick et al. 2016; Brady et al. 2014).  However specific aspects of PDMP laws maybe more useful than others. Buchmueller and Carey (2017) find that PDMPs which include a provision that the provider must access the PDMP before prescribing opioids are effective at reducing indicators of opioid abuse.  However, PDMPs without such a provision have no effect (Buchmueller and Carey 2017).

Data on marijuana laws come from the National Organization for the Reform of Marijuana Laws (NORML) (?State Laws? 2017).  There is some evidence that medical marijuana can be a substitute for opioid medication (Piper et al. 2017).  Powell et al. (2015) use the Treatment Episode Data Set and the National Vital Statistics System to find that when states allow marijuana dispensaries there is a decrease in opioid related overdose deaths and opioid addiction.  They find no evidence for the efficacy of medical marijuana laws in the absence of dispensaries (Powell, Pacula, and Jacobson 2015).

Data on state insurance rates, race, and state median income come from the Current Population Survey (U.S. Census Bureau 2016).  Four state control variables were created from this data.  A variable for the number of people uninsured divided by the total sampled, captures the percentage of people in the state without insurance.  Another variable records the percentage of the sample on Medicaid for each state.  A third variable is the percentage of the sample which is white.  Race is unlikely to change significantly overtime, but there is evidence that there is a large racial gap in opioid prescribing and addiction, so I have included it as a control (Johnson 2016).  The fourth variable is the state median income.  

Data on state population for the years 2005-2015 come from the American Community Survey (a survey conducted by the Census Bureau) (United States Census Bureau 2015).  Data for the years 2003-2004 come from an archived Census report (U.S. Census Bureau 2006).  

Table 1 shows the states that adopted the Medicaid expansion, that enacted must access laws for PDMPs, and that permit state medical marijuana dispensaries.  If the state did not adopt the legislation this is indicated by a blank.  If the state did adopt the legislation this is indicated by the date which it was adopted.  For the Medicaid expansion, if the state adopted legislation in January through July the legislation is counted as effective in the current year.  If the legislation was adopted in August through December, the legislation was counted as effective beginning in the subsequent year.   States for which the electoral college voted for a republican presidential candidate in 2016 are shown in red, and states which voted for a democratic candidate are shown in blue (Newman 2016).  The Medicaid expansion was more likely to occur in democratic states, but there were many republican states that also choose to expand.  

Table 2 presents summary statistics for the control variables and the outcome variables.  Some notable differences are that states that expand Medicaid at some point are more likely to have legal recreational marijuana, medical marijuana, state marijuana dispensaries, PDMPs, and PDMP must access laws.  The two groups are not very different in terms the demographic variables income, insurance, and race.  

\section{Methods}
With this data, I plan to use a difference in difference estimation technique. I create a variable called Expand, which equals 1 if the Medicaid expansion is in effect in that particular state and year and 0 otherwise.  The difference in difference equation to be estimated is:

\begin{equation}
    Y_{st}=\beta_0+\beta_1 Expand_{st}+\beta_2 X_{st} + \beta_3 State_s + \beta_4 Time_t + \epsilon_{st}
\end{equation}

Where $Y_{st}$ is the outcome variable (treatment admissions, deaths, or the number of grams sold of a particular drug) in a particular state in a particular year.  $Expand_{st}$  is a dummy variable which equals $1$ if the Medicaid expansion is in effect in a given state in a given year and $0$ otherwise.
$X_{st}$  is a set of control variables for each state in each year. $X_{st}$  includes population, median income, Medicaid rate, percent white, the legality of recreational marijuana, the legality of medical marijuana, state marijuana dispensaries, the presence of a PDMP, and the presence of a PDMP must access law.  $State_s$ is a set of dummy variables for each state. 
 $Time_t$ is as set of year dummy variables.  I cluster standard errors by state.

The difference in difference technique is ideal for a treatment-control experiment.  In a perfect world, states would be randomized into either expanding Medicaid or continuing as usual, and I would have data before and after the expansion occurred.  Then the treatment effect would be identified by the difference in difference estimator $\beta_1$. However, because of the political nature of the Medicaid expansion, the decision to expand Medicaid was not random.  If I am able to control for all the observable variables that may impact the decision to expand, and I can verify the parallel trends assumption outlined below, I can identify the treatment effect from equation (1).  

The parallel trends assumption is that in the absence of a treatment, treated and untreated states should have the same trend in the outcome variable.  To test this assumption, I both examine the pretends graphically and by using a regression. Figures 2-3 show the pretrends for the number of people on Medicaid per capita and uninsurance rate.  These were not outcome variables, but they do give a graphical representation of the effect that the Medicaid expansion had on insurance rates.  Particularly striking is that the trends in the uninsurance rate do not seem to be very different even after the Medicaid expansion (although uninsurance is much lower in expanding states).  Figures 4-9, show the pretrends for the all the outcome variables I consider.  The pretrends do not appear very different with the exception of fentanyl.

%need to decide if I am going to include the rest of this



\section{Results}
Table 4 shows the results from estimating equation (1).  All of the outcome variables are in per capita terms. Total overdose deaths and opioid related deaths increased significantly in states that expanded Medicaid.  Opioid related overdoses increased by .000016 percentage points (18$\%$ of the mean) and total overdose deaths increased by .00002 percentage points (43$\%$ of the mean).  However, these results should be interpreted with caution because there is some evidence that the parallel trends assumption was violated for overdose deaths.     

%need to put in results

\section{Conclusion}

Overall the results suggest that the Medicaid expansion does not affect indicators of opioid addiction.  The difference in difference model found that overdoses deaths increased in expanding states.  However, the parallel trends assumption was questionable so I favor the difference in difference in difference model in equation (3).  This model showed that they was no effect of the Medicaid expansion on opioid addiction.  This could be because there is actually no effect, or that there are two effects that work in opposite directions.  For example, more people may be prescribed opioids, which might lead to more addiction, but also increased access to preventative healthcare and addiction treatment may lessen addiction.  It is also possible that two years of post-treatment data is not enough to affect something like addiction which occurs due to sustained use over a long period of time.  Further research is needed on this topic.  I would like to extend this analysis to use the data on opioid addiction treatment admissions and other indicators of addiction.  As it stands, this paper suggests that there should not be a concern about an increase in health insurance leading to an increase in addiction. 

\bibliographystyle{plainnat}
\bibliography{opioidreferences}



\end{document}

